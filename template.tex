\ctexset{
  section={
    name={第,章},
    number =  \fontspec{SimSun} \arabic{section},
    format += \bfseries \centering \zihao{3}
   },
  subsection={
  number = \fontspec{SimSun} \arabic{section}.\arabic{subsection} ,
  format += \bfseries \zihao{4} \fontspec{SimSun},
  indent = 2em,
  aftername = \hspace{5pt},
  beforeskip = 2ex 
  },
  subsubsection={
  number =  \fontspec{SimSun} \arabic{section}.\arabic{subsection}.\arabic{subsubsection} ,
  format += \bfseries \zihao{-4} \fontspec{SimSun} ,
  indent = 2em,
  aftername = \hspace{5pt},
  beforeskip = 2ex,
  fixskip = true
  }
  }



\usepackage{ctex}
\usepackage{fontspec} % 字体设置
\usepackage[a4paper, left=3.17cm, right=3.17cm, top=2.54cm, bottom=2.54cm]{geometry} % 页面设置
\usepackage{titlesec} % 标题设置
\usepackage{setspace}  % 设置行间
\usepackage{ifthen} %if-then-else、equal
\usepackage{indentfirst} %首行缩进
\usepackage{tocloft} %自定义目录
\usepackage{xeCJK}
\usepackage{titletoc}
\usepackage{algorithmicx}
\usepackage[hidelinks]{hyperref} %目录跳转
\usepackage{graphicx}
\usepackage{subcaption}
% \usepackage{showframe}

% 页眉设置fancy包必须在geometry后面不然页眉不居中
\usepackage{fancyhdr}
\pagestyle{fancy}
\fancyhf{} % 清空当前的页眉和页脚设置否则会使用plain样式
\renewcommand{\headrulewidth}{0.8pt} % 设置页眉线宽度
\chead{\zihao{5} \fontspec{SimSun} \textbf{长春理工大学本科生毕业设计}} % 页眉内容
\cfoot{\thepage}



% 设置目录样式
\renewcommand{\contentsname}{\hfill{\zihao{3} \fontspec{SimSun} \bf{目录} }\hfill} % 设置目录标题字号且居中

\renewcommand{\cftsecfont}{\zihao{-4} \songti} % 设置章节标题字体为宋体小四 
\renewcommand{\cftsubsecfont}{\zihao{-4} \songti} % 设置章节标题字体为宋体小四


% 目录section标题设置

\titlecontents{section}[0pt]{\linespread{1.5}\filright}
{\contentspush{\thecontentslabel\
  }}
{}{\titlerule*[3pt]{.}\contentspage}

% 目录subsection标题设置
\titlecontents{subsection}[15pt]{\linespread{1.5}\filright}
{\contentspush{\thecontentslabel\
  }}
{}{\titlerule*[3pt]{.}\contentspage}

\titlecontents{subsubsection}[30pt]{\linespread{1.5}\filright}
{\contentspush{\thecontentslabel\
  }}
{}{\titlerule*[3pt]{.}\contentspage}
% \setlength{\cftbeforesecskip}{1.25} % 设置章节标题行间距
% 设置英文摘要标题样式
\newcommand{\englishAbstractSection}{%
  \section*{\centering\bf{\zihao{3}{\fontspec{Times New Roman} ABSTRACT}}}
  \vspace{1\baselineskip}
}

% 设置中文摘要标题样式
\newcommand{\chineseAbstractSection}{%
  \section*{\centering\textbf{\zihao{3}{\fontspec{SimSun} 摘要}}}
  \vspace{1\baselineskip}
}

% \renewcommand{\subsection}[1]{\textbf{\zihao{4} #1}}

% 定义一个名为 'onequaters' 的环境
\newenvironment{onequaters}
{\begin{spacing}{1.5}\zihao{-4}}  % 在环境开始时设置行距为1.25倍和字号为小四
    {\end{spacing}}  % 在环境结束时恢复默认行距和字号

% 英文字体
\setmainfont{Times New Roman} % 常规英文字体
\setsansfont{Arial} % 非衬线字体
\setmonofont{Consolas} % 打字机字体,代码字体


% 中文字体配置部分
\setCJKmainfont[AutoFakeBold=2]{SimSun}%正文字体
\setCJKsansfont[AutoFakeBold=3]{楷体}%无衬线字体
\setCJKmonofont[AutoFakeBold=3]{SimHei}%等宽字体
% 使用 AutoFakeBold 可以实现「伪粗体」,和word中相同。
 

\captionsetup[subfigure]{labelformat=brace, labelsep=space} % 子图a)效果加空格
\DeclareCaptionLabelFormat{figureLableFormat}{{\zihao{-5} 图 \thesection.\arabic{figure}}}

\captionsetup[figure]{labelformat=figureLableFormat}
