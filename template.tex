\ctexset{
  section={
    name={第,章},
    number =  \fontspec{SimSun} \arabic{section},
    format += \bfseries \centering \zihao{3} \setcounter{figure}{0} \setcounter{table}{0} \setcounter{equation}{0}
   },
  subsection={
  number = \fontspec{SimSun} \arabic{section}.\arabic{subsection} ,
  format += \bfseries \zihao{4} \fontspec{SimSun},
  indent = 2em,
  aftername = \hspace{5pt},
  beforeskip = 2ex,
  break = \Needspace{.5\textheight}
  },
  subsubsection={
  number =  \fontspec{SimSun} \arabic{section}.\arabic{subsection}.\arabic{subsubsection} ,
  format += \bfseries \zihao{-4} \fontspec{SimSun} ,
  indent = 2em,
  aftername = \hspace{5pt},
  beforeskip = 2ex,
  }
}



\usepackage{ctex}
\usepackage{fontspec} % 字体设置
\usepackage[a4paper, left=3.17cm, right=3.17cm, top=2.54cm, bottom=2.54cm]{geometry} % 页面设置
\usepackage{titlesec} % 标题设置
\usepackage{setspace}  % 设置行间
\usepackage{ifthen} %if-then-else、equal
\usepackage{indentfirst} %首行缩进
\usepackage{tocloft} %自定义目录
\usepackage{xeCJK}
\usepackage{titletoc}
\usepackage{algorithmicx}
\usepackage[hidelinks, backref]{hyperref} %目录、文献跳转
\usepackage{graphicx}
\usepackage{caption}
\usepackage{subcaption}
\usepackage{amsmath} % 加载公式宏包
\usepackage{cleveref}
\usepackage{booktabs} % 表格粗体线
\usepackage{makecell} % 表格内换行
\usepackage{mathtools}
\usepackage{enumitem} % 续航序列
\usepackage[ruled,vlined,linesnumbered]{algorithm2e}
\usepackage{gbt7714}

\bibliographystyle{gbt7714-numerical} % 采用GBT-7714格式

% \usepackage{showframe}

% 页眉设置fancy包必须在geometry后面不然页眉不居中
\usepackage{fancyhdr}
\pagestyle{fancy}
\fancyhf{} % 清空当前的页眉和页脚设置否则会使用plain样式
\renewcommand{\headrulewidth}{0.8pt} % 设置页眉线宽度
\chead{\zihao{5} \fontspec{SimSun} \textbf{长春理工大学本科生毕业设计}} % 页眉内容
\cfoot{\thepage}



% 设置目录样式
\renewcommand{\contentsname}{\hfill{\zihao{3} \fontspec{SimSun} \bf{目录} }\hfill} % 设置目录标题字号且居中

\renewcommand{\cftsecfont}{\zihao{-4} \songti} % 设置章节标题字体为宋体小四 
\renewcommand{\cftsubsecfont}{\zihao{-4} \songti} % 设置章节标题字体为宋体小四
\renewcommand{\cftsubsubsecfont}{\zihao{-4} \songti}

% 目录section标题设置

\titlecontents{section}[0pt]{\filright}
{\contentspush{\thecontentslabel\
  }}
{}{\titlerule*[3pt]{.}\contentspage}

% 目录subsection标题设置
\titlecontents{subsection}[15pt]{\filright}
{\contentspush{\thecontentslabel\
  }}
{}{\titlerule*[3pt]{.}\contentspage}

\titlecontents{subsubsection}[30pt]{\filright}
{\contentspush{\thecontentslabel\
  }}
{}{\titlerule*[3pt]{.}\contentspage}


\newcommand{\sectionbreak}{\clearpage} % 该section内容结束之后自动分页

% 设置英文摘要标题样式
\newcommand{\englishAbstractSection}{%
  \section*{\centering\bf{\zihao{3}{\fontspec{Times New Roman} ABSTRACT}}}
  \vspace{0.5\baselineskip}
}

% 设置中文摘要标题样式
\newcommand{\chineseAbstractSection}{%
  \section*{\centering\textbf{\zihao{3}{\fontspec{SimSun} 摘要}}}
  \vspace{0.5\baselineskip}
}

% 定义一个名为 'onequaters' 的环境
\newenvironment{onequaters}
{\begin{spacing}{1.5}\zihao{-4}}  % 在环境开始时设置行距为1.25倍和字号为小四
    {\end{spacing}}  % 在环境结束时恢复默认行距和字号

% 英文字体
\setmainfont{Times New Roman} % 常规英文字体
\setsansfont{Arial} % 非衬线字体
\setmonofont{Consolas} % 打字机字体,代码字体

\setmathrm{LMRoman10-Regular}

% 中文字体配置部分
\setCJKmainfont[AutoFakeBold=2]{SimSun}%正文字体
\setCJKsansfont[AutoFakeBold=3]{楷体}%无衬线字体
\setCJKmonofont[AutoFakeBold=3]{SimHei}%等宽字体
% 使用 AutoFakeBold 可以实现「伪粗体」,和word中相同。
 

\DeclareCaptionLabelFormat{figureLableFormat}{\zihao{5}{图\thesection.\arabic{figure}}}
\DeclareCaptionLabelFormat{tableLabelFormat}{\zihao{5}{表\thesection.\arabic{table}}}
\DeclareCaptionFormat{figureFormat}{\zihao{5}#1#2#3} % label sep及caption本身字号设置
\captionsetup[table]{labelformat=tableLabelFormat, labelsep=space}


\captionsetup[subfigure]{labelformat=brace, labelsep=space, format=figureFormat} % 子图a)效果加空格
\captionsetup[figure]{labelformat=figureLableFormat, labelsep=space, format=figureFormat}

\crefformat{figure}{{图\thesection.#2#1#3}}
\crefformat{table}{{表\thesection.#2#1#3}}
\crefformat{equation}{式(#2#1#3)}

\renewcommand{\theequation}{\thesection-\arabic{equation}} % 公式编号覆写


\setlist[itemize]{
  before=\setstretch{1.0}\zihao{-4},
  after=\vspace{-\baselineskip},
}

% 设置enumerate环境的字体和行间距
\setlist[enumerate]{
  before=\setstretch{1.0}\zihao{-4},
  after=\vspace{-\baselineskip},
  label=\arabic*)
}

% 算法包语言设置
\SetKwInput{KwData}{输入}
\SetKwInput{KwResult}{输出}
\renewcommand{\algorithmcfname}{算法}